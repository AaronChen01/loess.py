% TEMPLATE for Usenix papers, specifically to meet requirements of
%  USENIX '05
% originally a template for producing IEEE-format articles using LaTeX.
%   written by Matthew Ward, CS Department, Worcester Polytechnic Institute.
% adapted by David Beazley for his excellent SWIG paper in Proceedings,
%   Tcl 96
% turned into a smartass generic template by De Clarke, with thanks to
%   both the above pioneers
% use at your own risk.  Complaints to /dev/null.
% make it two column with no page numbering, default is 10 point

% Munged by Fred Douglis <douglis@research.att.com> 10/97 to separate
% the .sty file from the LaTeX source template, so that people can
% more easily include the .sty file into an existing document.  Also
% changed to more closely follow the style guidelines as represented
% by the Word sample file. 

% Note that since 2010, USENIX does not require endnotes. If you want
% foot of page notes, don't include the endnotes package in the 
% usepackage command, below.

% This version uses the latex2e styles, not the very ancient 2.09 stuff.
\documentclass[letterpaper,twocolumn,10pt]{article}
\usepackage{usenix,epsfig,endnotes}
\begin{document}

%don't want date printed
\date{}

%make title bold and 14 pt font (Latex default is non-bold, 16 pt)
\title{
  \Large \bf Forecasting Attributes of Tropical Cyclones\\
             Using Robust Locally Weighted Regression
}

\author {
  {\rm Ali Yahya}\\
  Stanford University
  % copy the following lines to add more authors
  % \and
  % {\rm Name}\\
  %Name Institution
} % end author

\maketitle

% Use the following at camera-ready time to suppress page numbers.
% Comment it out when you first submit the paper for review.
% \thispagestyle{empty}


\subsection*{Abstract}
Your Abstract Text Goes Here.  Just a few facts.
Whet our appetites.

\section{Introduction}

Tropical cyclones are highly structured storm systems that are characterized by
numerous thunderstorms, strong winds, and heavy rain that revolve around a large
low-pressure center. Because of the high degree of structure in tropical
cyclones, their data model is well defined and relatively self-contained.
Consequently, forecast models that treat tropical cyclones as systems in
isolation---that is, without consideration for collateral or larger scale
factors---can be more successful than similar models applied to less structured
storm systems.

\subsection{Patterns in Hurricane Data Models}
At any given moment, any one tropical cyclone and its present behavior can be
almost entirely described with values for a relatively small set of attributes.
One important subset of those attributes could include: minimum central
pressure, maximum wind speeds, latitude, longitude, radius of eyewall, radius of
maximum wind speed, radius of outer closed isobar, pressure of outer closed
isobar, among others. Moreover, previous studies have indicated that said
attributes not only represent a reasonably complete snapshot of a tropical
cyclone's behavior but also that they are highly correlated with each
other.~\cite{SHIPS}

In this paper, we analyze data for approximately 10,000 hurricanes in the
Atlantic and Eastern North Pacific Basins and observe a number of statistically
significant correlations between seven attributes pertaining to tropical
cyclones.

Further, we formalize the aforementioned observations and present a forecast
model, based on locally weighted linear regression, that allows meteorologists
to use historical storm data to extrapolate the value of a single corrupted data
attribute in the data model of a storm, given correct values for a sufficient
number of other attributes.

\subsection{Motivation:\\Extrapolating Sensor Data}
Often times, at the most critical moment of a storm's development, sensor data
about its state can be incomplete or fraught with inconsistencies. In situations
like that, a forecast model that is based on a fundamental understanding of the
relationships between a storm's different attributes can help verify or 
reconstruct a dataset of an ongoing storm.

Additionally, a forecast model that sheds light on the interactions between
the different attributes in a storm's data model, enables meteorologists to
better predict the side-effects of changes to any one of those attributes.
Consequently, a successful forecast model for tropical cyclones may not only
improve the accuracy of warnings issued, but also their qualitative preciseness.

For instance, although it's fairly well understood that a increase in minimum
central pressure will result in a decrease of maximum wind speeds, it is also
important to consider a more subtle effect: A decrease in minimum central
pressure will indeed result in lower wind speeds, but because of the law of
conservation of angular momentum, the radius around which maximum wind speeds
revolve will also increase. Therefore, the resulting storm surge and size of the
area of devastation will be different than with lower pressure and higher wind
speeds. Understanding exactly how such factors are likely to be different is
critical for the effectiveness of a tropical cyclone warning system.


\section{Patterns Observed}

As a preliminary step to crafting a unified forecast model, we present a number
of statistically significant pair-wise correlations that were observed from
plots of data from \emph{The Tropical Cyclone Extended Best Track
Dataset}.~\cite{BestTrackDataset} The following attributes of approximately 
10,000 tropical cyclones were observed:
\begin{itemize}
  \item latitude
  \item longitude
  \item maximum wind speed
  \item minimum central pressure
  \item radius of maximum wind speed
  \item pressure of outer closed isobar
  \item radius of outer closed isobar
\end{itemize}


\bibliographystyle{acm}
\bibliography{references}

% \theendnotes

\end{document}
